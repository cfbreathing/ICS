\documentclass[20pt]{ctexart}
\usepackage{graphicx}
\usepackage{amsmath}
\usepackage{url}
\usepackage{subfigure}
\usepackage{float}
\usepackage{lmodern}
\usepackage{xeCJK}
\usepackage{tikz}
\usepackage{enumitem}
\usepackage{verbatimbox}
\usepackage{cite}
\usepackage{amsfonts}
\usepackage{amsthm}
\usepackage{geometry}
\usepackage{verbatimbox}
\usepackage{caption}
\usepackage{listings}
\usepackage[ruled,linesnumbered]{algorithm2e}

%设置新环境
\newtheorem{example}{例}             
\newtheorem{theorem}{定理}[section] 
\newtheorem{definition}{定义}[section]
\newtheorem{property}{性质}
\newtheorem{proposition}{命题}
\newtheorem{lemma}{引理}
\newtheorem{corollary}{推论}
\newtheorem{remark}{注}
\newtheorem{condition}{条件}
\newtheorem{conclusion}{结论}
\newtheorem{assumption}{假设}
\newenvironment{solution}{\begin{proof}[\indent\bf 解]}{\end{proof}}%设置新环境
\CTEXsetup[format={\Large\bfseries}]{section}%加入字体
\geometry{a4paper,scale=0.8}%设置文档格式
\tikzstyle{file} = [rectangle, rounded corners, minimum width = 3cm, minimum height=1.2cm ,text centered, draw = black]%设置流程图
\tikzstyle{dots} = [rectangle, rounded corners, minimum width = 1.5cm, minimum height=2cm ,text centered, draw = black,text width=3cm]%设置流程图
\tikzstyle{arrow} = [->,>=stealth]%设置流程图
\usetikzlibrary{arrows, decorations.pathmorphing, backgrounds, positioning, fit, petri, automata}%使用流程图元素
\bibliographystyle{unsrt}%设置引用格式

\title{Homework1}
\author{作者:罗文杰\\专业: 计算机科学与技术\\学号: 3210102456}
\date{}

\begin{document}
\maketitle

\section*{1.9}
Yes, they are, provided they can be organised into a series of unambiguous logical statements.

\section*{1.11}
Open a folder, find the file, and do something.

\section*{1.16}
Storage system, data types, register, addressing modes.

\section*{1.18}
A single microarchitecture usually implements only one ISA, but a ISA can have many microarchitecture to exist for it.

\section*{2.8}
(a) binary: 0111 1111, decimal: 127.

(b) binary: 1000 0000, decimal: -128.

(c) $2^{n-1}-1 $

(d) $-2^{n-1}$

\section*{2.14}
(a) 1100

(b) 1010

(c) 1111

(d) 01011

(e) 10000

\section*{2.22}
16384+16384=-32768

\section*{2.24}
32768+32768=0

\section*{2.27}
The problem is that the two positive number adds to a nagetive number because overflow.

\section*{2.34}
(a) 0111

(b) 0111

(c) 1101

(d) 0110

\bibliography{reference}
\end{document}
