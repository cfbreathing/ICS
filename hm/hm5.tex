\documentclass[20pt]{ctexart}
\usepackage{graphicx}
\usepackage{amsmath}
\usepackage{url}
\usepackage{subfigure}
\usepackage{float}
\usepackage{lmodern}
\usepackage{xeCJK}
\usepackage{tikz}
\usepackage{enumitem}
\usepackage{verbatimbox}
\usepackage{cite}
\usepackage{amsfonts}
\usepackage{amsthm}
\usepackage{geometry}
\usepackage{verbatimbox}
\usepackage{caption}
\usepackage{listings}
\usepackage[ruled,linesnumbered]{algorithm2e}

%设置新环境
\newtheorem{example}{例}             
\newtheorem{theorem}{定理}[section] 
\newtheorem{definition}{定义}[section]
\newtheorem{property}{性质}
\newtheorem{proposition}{命题}
\newtheorem{lemma}{引理}
\newtheorem{corollary}{推论}
\newtheorem{remark}{注}
\newtheorem{condition}{条件}
\newtheorem{conclusion}{结论}
\newtheorem{assumption}{假设}
\newenvironment{solution}{\begin{proof}[\indent\bf 解]}{\end{proof}}%设置新环境
\CTEXsetup[format={\Large\bfseries}]{section}%加入字体
\geometry{a4paper,scale=0.8}%设置文档格式
\tikzstyle{file} = [rectangle, rounded corners, minimum width = 3cm, minimum height=1.2cm ,text centered, draw = black]%设置流程图
\tikzstyle{dots} = [rectangle, rounded corners, minimum width = 1.5cm, minimum height=2cm ,text centered, draw = black,text width=3cm]%设置流程图
\tikzstyle{arrow} = [->,>=stealth]%设置流程图
\usetikzlibrary{arrows, decorations.pathmorphing, backgrounds, positioning, fit, petri, automata}%使用流程图元素
\bibliographystyle{unsrt}%设置引用格式

\title{Homework5}
\author{作者:罗文杰\\专业: 计算机科学与技术\\学号: 3210102456}
\date{}

\begin{document}
\maketitle

\section*{5.15}
R1:0x3121

R2:0x4566

R3:0xabcd

R4:0xabcd

\section*{5.16}
(a) PC-relative mode

(b) Indirect mode

(c) Base-offset mode

\section*{5.37}
IR, PC, Reg File, the SEXT unit, Memory, MDR, MAR, is connected to IR[8:0]; MAXMUX and GateMARMUX implement the LDI instruction, alongwith NZP and the logic which goes with it.

\section*{5.39}
IR, PC, Reg File, the SEXT unit, Memory, MDR, MAR, is connected to IR[8:0]; MAXMUX and GateMARMUX implement the LEA instruction, alongwith NZP and the logic which goes with it.

\section*{5.50}
The conditional Branch: PC.

The Load Effective Address: R7.

The LD instruction: R7.
\section*{6.9}
0010 000 0 0000 0101; LD, R0, Z 

0010 001 0 0000 0101; LD, R1, C 

1111 0000 0010 0001; L, TRAP x21 

0001 001 001 1 11111; ADD R1, R1, \#-1 

0000 001 1 11111101; BRp PC-3

1111 0000 0010 0101; TRAP x25 

0000 0000 0101 1010; Z .FILL x5A 

0000 0000 0110 0100; C .FILL \#100 


\section*{6.10}
This program branches Brz to x3010 if the number in R2 is even and branches to x3011 if the number in R2 is odd.

x3000 0101 000 010 1 00001; AND R0, R2, \#1 

x3001 0000 010 0 00001110; BRp Even 

x3002 0000 101 0 00001111; BRp Odd 

\bibliography{reference}
\end{document}
