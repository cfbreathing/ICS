\documentclass[20pt]{ctexart}
\usepackage{graphicx}
\usepackage{amsmath}
\usepackage{url}
\usepackage{subfigure}
\usepackage{float}
\usepackage{lmodern}
\usepackage{xeCJK}
\usepackage{tikz}
\usepackage{enumitem}
\usepackage{verbatimbox}
\usepackage{cite}
\usepackage{amsfonts}
\usepackage{amsthm}
\usepackage{geometry}
\usepackage{verbatimbox}
\usepackage{caption}
\usepackage{listings}
\usepackage[ruled,linesnumbered]{algorithm2e}

%设置新环境
\newtheorem{example}{例}             
\newtheorem{theorem}{定理}[section] 
\newtheorem{definition}{定义}[section]
\newtheorem{property}{性质}
\newtheorem{proposition}{命题}
\newtheorem{lemma}{引理}
\newtheorem{corollary}{推论}
\newtheorem{remark}{注}
\newtheorem{condition}{条件}
\newtheorem{conclusion}{结论}
\newtheorem{assumption}{假设}
\newenvironment{solution}{\begin{proof}[\indent\bf 解]}{\end{proof}}%设置新环境
\CTEXsetup[format={\Large\bfseries}]{section}%加入字体
\geometry{a4paper,scale=0.8}%设置文档格式
\tikzstyle{file} = [rectangle, rounded corners, minimum width = 3cm, minimum height=1.2cm ,text centered, draw = black]%设置流程图
\tikzstyle{dots} = [rectangle, rounded corners, minimum width = 1.5cm, minimum height=2cm ,text centered, draw = black,text width=3cm]%设置流程图
\tikzstyle{arrow} = [->,>=stealth]%设置流程图
\usetikzlibrary{arrows, decorations.pathmorphing, backgrounds, positioning, fit, petri, automata}%使用流程图元素
\bibliographystyle{unsrt}%设置引用格式

\title{Homework3}
\author{作者:罗文杰\\专业: 计算机科学与技术\\学号: 3210102456}
\date{}

\begin{document}
\maketitle

\section*{3.38}
The memory address is the location of data, and the memory’s addressability is the number of bits that one memory location can store.

\section*{3.40}
(a) 4 locations.

(b) 4 bits.

(c) 0001.

\section*{3.53}
(a)\begin{table}[H]
  \centering
  \begin{tabular}{|c|c|c|c|c|c|c|c|}
      \hline
      cycle0 & cycle1 & cycle2 & cycle3 & cycle4 & cycle5 & cycle6 & cycle7  \\
      \hline
      0 & 1 & 1 & 1 & 1 & 0 & 0 & 0 \\
      \hline
      0 & 1 & 1 & 0 & 0 & 1 & 1 & 0 \\
      \hline  
      0 & 1 & 0 & 1 & 0 & 1 & 0 & 1 \\
      \hline  
      \end{tabular}
      \end{table}

Delaying the clock cycle.
\section*{3.61}
(a)\begin{table}[H]
  \centering
  \begin{tabular}{|c|c|c|c|c|c|}
      \hline
      S1 & S0 & X & Z & S1' & S0' \\
      \hline
      0 & 0 & 0 & 1 & 0 & 0 \\
      \hline
      0 & 0 & 1 & 1 & 0 & 1 \\
      \hline  
      0 & 1 & 0 & 0 & 1 & 0 \\
      \hline  
      0 & 1 & 1 & 0 & 0 & 0 \\
      \hline 
      1 & 0 & 0 & 0 & 0 & 1 \\
      \hline
      1 & 0 & 1 & 0 & 1 & 0 \\
      \hline
      1 & 1 & 0 & 0 & 0 & 0 \\
      \hline
      1 & 1 & 1 & 0 & 0 & 0 \\
      \hline
      \end{tabular}
      \end{table}

(b)\begin{figure}[H]
  \centering
  \includegraphics[width=0.7\textwidth]{6.jpg}
\end{figure}

\section*{4.1}
Memory:storing data.

Processing Unit:process information

Input:take information into the computer

Output:take information out of the computer.

Control Unit:control other parts to make them perform correctly.

\section*{4.7}
60 opcodes need 6bits, and 32 registers need 5bits, so IMM has 16 bits, the rang of its value is (-32768,32767).

\bibliography{reference}
\end{document}
