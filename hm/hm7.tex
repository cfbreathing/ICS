\documentclass[20pt]{ctexart}
\usepackage{graphicx}
\usepackage{amsmath}
\usepackage{url}
\usepackage{subfigure}
\usepackage{float}
\usepackage{lmodern}
\usepackage{xeCJK}
\usepackage{tikz}
\usepackage{enumitem}
\usepackage{verbatimbox}
\usepackage{cite}
\usepackage{amsfonts}
\usepackage{amsthm}
\usepackage{geometry}
\usepackage{verbatimbox}
\usepackage{caption}
\usepackage{listings}
\usepackage[ruled,linesnumbered]{algorithm2e}

%设置新环境
\newtheorem{example}{例}             
\newtheorem{theorem}{定理}[section] 
\newtheorem{definition}{定义}[section]
\newtheorem{property}{性质}
\newtheorem{proposition}{命题}
\newtheorem{lemma}{引理}
\newtheorem{corollary}{推论}
\newtheorem{remark}{注}
\newtheorem{condition}{条件}
\newtheorem{conclusion}{结论}
\newtheorem{assumption}{假设}
\newenvironment{solution}{\begin{proof}[\indent\bf 解]}{\end{proof}}%设置新环境
\CTEXsetup[format={\Large\bfseries}]{section}%加入字体
\geometry{a4paper,scale=0.8}%设置文档格式
\tikzstyle{file} = [rectangle, rounded corners, minimum width = 3cm, minimum height=1.2cm ,text centered, draw = black]%设置流程图
\tikzstyle{dots} = [rectangle, rounded corners, minimum width = 1.5cm, minimum height=2cm ,text centered, draw = black,text width=3cm]%设置流程图
\tikzstyle{arrow} = [->,>=stealth]%设置流程图
\usetikzlibrary{arrows, decorations.pathmorphing, backgrounds, positioning, fit, petri, automata}%使用流程图元素
\bibliographystyle{unsrt}%设置引用格式

\title{Homework7}
\author{作者:罗文杰\\专业: 计算机科学与技术\\学号: 3210102456}
\date{}

\begin{document}
\maketitle

\section*{8.1}
Last in first out.

\section*{8.7}
\begin{verbatim}
    ; R4 is the address of output
    ; R3 is the number of elements needed to pop
    POP     ST R2, Save2 
            ST R1, Save1
            ST R0, Save0
            LD R1, BASE  
            ADD R2, R6, R1 
            BRz fail_exit 
            ADD R0, R4, #0
            ADD R1, R3, #0
            ADD R5, R6, R3
            ADD R5, R5, #-1
            ADD R6, R6, R3
    poloop  LDR R2, R5, #0
            STR R2, R0, #0
            ADD R0, R0, #1
            ADD R5, R5, #-1
            ADD R1, R1, #-1
            BRp pop_loop
            BRnzp success_exit
    ; R4 is the address of input
    ; R3 is the number of elements needed to push
    PUSH    ST R2, Save2 
            ST R1, Save1 
            ST R0, Save0
            LD R1,MAX 
            ADD R2,R6,R1 
            BRz fail_exit 
            ADD R0, R4, #0
            ADD R1, R3, #0
            ADD R5, R6, #-1
            NOT R2, R3
            ADD R2, R2, #1
            ADD R6, R6, R2
    puloop  LDR R2, R0, #0
            STR R2, R5, #0
            ADD R0, R0, #1
            ADD R5, R5, #-1
            ADD R1, R1, #-1
            BRp push_loop
    success_exit   LD R0, Save0
            LD R1, Save1 
            LD R2, Save2 
            AND R5, R5, #0 
            RET
    fail_exit   LD R0, Save0
            LD R1, Save1 
            LD R2, Save2 
            AND R5, R5, #0
            ADD R5, R5, #1
            RET
    BASE    .FILL xC000 
            MAX .FILL xC005
            Save0 .FILL x0000
            Save1 .FILL x0000
            Save2 .FILL x0000
\end{verbatim}

\section*{8.8}
(a)\begin{figure}[H]
    \centering
    \includegraphics[width=0.3\textwidth]{img/4.png}
\end{figure}

(b) After PUSH J and PUSH K operations.

(c) A F M

\section*{8.12}
\begin{verbatim}
        .ORIG   x3050
    A   .FILL   x41 ; ASCII A
        .FILL   B
        .FILL   B ; connected areas
        .FILL   D
        .FILL   E
        .FILL   E
        .FILL   x0

    B   .FILL   x42 ; ASCII B
        .FILL   C
        .FILL   A ; connected areas
        .FILL   D
        .FILL   C
        .FILL   C
        .FILL   x0
        
    C   .FILL   x43 ; ASCII C
        .FILL   B
        .FILL   B ; connected areas
        .FILL   B
        .FILL   D
        .FILL   x0
        
    D   .FILL   x44 ; ASCII D
        .FILL   E
        .FILL   A ; connected areas
        .FILL   B
        .FILL   C
        .FILL   E
        .FILL   x0

    E   .FILL   x45 ; ASCII E
        .FILL   A
        .FILL   A ; connected areas
        .FILL   A
        .FILL   D
        .FILL   x0
\end{verbatim}

\section*{8.14}
(a) JSR X

(b) R1, R3, \#1

(c) R2, R4, \#1

(d) R1, R2

(e) ADD R0, R0, R1

(f) R5, \#1

(g) LABEL

(h) ADDING

(i) ADDING

\end{document}
