\documentclass[20pt]{ctexart}
\usepackage{graphicx}
\usepackage{amsmath}
\usepackage{url}
\usepackage{subfigure}
\usepackage{float}
\usepackage{lmodern}
\usepackage{xeCJK}
\usepackage{tikz}
\usepackage{enumitem}
\usepackage{verbatimbox}
\usepackage{cite}
\usepackage{amsfonts}
\usepackage{amsthm}
\usepackage{geometry}
\usepackage{verbatimbox}
\usepackage{caption}
\usepackage{listings}
\usepackage[ruled,linesnumbered]{algorithm2e}

%设置新环境
\newtheorem{example}{例}             
\newtheorem{theorem}{定理}[section] 
\newtheorem{definition}{定义}[section]
\newtheorem{property}{性质}
\newtheorem{proposition}{命题}
\newtheorem{lemma}{引理}
\newtheorem{corollary}{推论}
\newtheorem{remark}{注}
\newtheorem{condition}{条件}
\newtheorem{conclusion}{结论}
\newtheorem{assumption}{假设}
\newenvironment{solution}{\begin{proof}[\indent\bf 解]}{\end{proof}}%设置新环境
\CTEXsetup[format={\Large\bfseries}]{section}%加入字体
\geometry{a4paper,scale=0.8}%设置文档格式
\tikzstyle{file} = [rectangle, rounded corners, minimum width = 3cm, minimum height=1.2cm ,text centered, draw = black]%设置流程图
\tikzstyle{dots} = [rectangle, rounded corners, minimum width = 1.5cm, minimum height=2cm ,text centered, draw = black,text width=3cm]%设置流程图
\tikzstyle{arrow} = [->,>=stealth]%设置流程图
\usetikzlibrary{arrows, decorations.pathmorphing, backgrounds, positioning, fit, petri, automata}%使用流程图元素
\bibliographystyle{unsrt}%设置引用格式

\title{Homework2}
\author{作者:罗文杰\\专业: 计算机科学与技术\\学号: 3210102456}
\date{}

\begin{document}
\maketitle

\section*{2.40}
(a) 2

(b) -17

(c) $+\infty$

(d) -3.125

\section*{2.43}
(a) Hello!

(b) hELLO!

(c) Computers!

(d) LC-2

\section*{3.6}
\begin{table}[H]
\centering
\begin{tabular}{lr|lcr}
    \hline
    A & B & C & D & Z \\
    \hline
    0 & 0 & 1 & 1 & 0 \\
    0 & 1 & 1 & 0 & 0 \\
    1 & 0 & 0 & 1 & 0 \\
    1 & 1 & 0 & 0 & 1 \\
    \hline
    \end{tabular}
    \end{table}

\section*{3.20}
The 16-input multiplexer have 4 select lines and 1 output line.

\section*{3.23}
(a)\begin{table}[H]
\centering
\begin{tabular}{lcr|lr}
    \hline
    A & B & C & D & Z \\
    \hline
    0 & 0 & 0 & 0 & 0 \\
    0 & 0 & 0 & 1 & 0 \\
    0 & 0 & 1 & 0 & 0 \\
    0 & 0 & 1 & 1 & 0 \\
    0 & 1 & 0 & 0 & 0 \\
    0 & 1 & 0 & 1 & 0 \\
    0 & 1 & 1 & 0 & 0 \\
    0 & 1 & 1 & 1 & 0 \\
    1 & 0 & 0 & 0 & 1 \\
    1 & 0 & 0 & 1 & 1 \\
    1 & 0 & 1 & 0 & 1 \\
    1 & 0 & 1 & 1 & 1 \\
    1 & 1 & 0 & 0 & 1 \\
    1 & 1 & 0 & 1 & 1 \\
    1 & 1 & 1 & 0 & 1 \\
    1 & 1 & 1 & 1 & 1 \\
    \hline
    \end{tabular}
    \end{table}

It has 16 rows.

(b) \begin{figure}[H]
    \centering
    \includegraphics[width=0.5\textwidth]{1.png}
  \end{figure}

\section*{3.26}
\begin{figure}[H]
    \centering
    \includegraphics[width=0.5\textwidth]{4.jpg}
    \caption*{gate-level logic}
  \end{figure}

\begin{figure}[H]
    \centering
    \includegraphics[width=0.5\textwidth]{5.jpg}
    \caption*{transistor diagram}
  \end{figure}
\section*{3.30}
(a) If X = 0,then S = A + B. And if X = 1,then S = A + B.

(b)Take a same construction as the Figure 3.42, but make C = (NOT)B, and make Carry in = X.

\section*{3.36}
(a)\begin{table}[H]
    \centering
    \begin{tabular}{lr|lcr}
    \hline
    A & B & G & E & L \\
    \hline
    0 & 0 & 0 & 1 & 0 \\
    0 & 1 & 0 & 0 & 1 \\
    1 & 0 & 1 & 0 & 0 \\
    1 & 1 & 0 & 1 & 0 \\
    \hline
    \end{tabular}
    \end{table}

(b) \begin{figure}[H]
    \centering
    \includegraphics[width=0.5\textwidth]{2.png}
  \end{figure}

(c) \begin{figure}[H]
    \centering
    \includegraphics[width=0.5\textwidth]{3.png}
  \end{figure}

\bibliography{reference}
\end{document}
