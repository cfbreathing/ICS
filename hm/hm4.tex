\documentclass[20pt]{ctexart}
\usepackage{graphicx}
\usepackage{amsmath}
\usepackage{url}
\usepackage{subfigure}
\usepackage{float}
\usepackage{lmodern}
\usepackage{xeCJK}
\usepackage{tikz}
\usepackage{enumitem}
\usepackage{verbatimbox}
\usepackage{cite}
\usepackage{amsfonts}
\usepackage{amsthm}
\usepackage{geometry}
\usepackage{verbatimbox}
\usepackage{caption}
\usepackage{listings}
\usepackage[ruled,linesnumbered]{algorithm2e}

%设置新环境
\newtheorem{example}{例}             
\newtheorem{theorem}{定理}[section] 
\newtheorem{definition}{定义}[section]
\newtheorem{property}{性质}
\newtheorem{proposition}{命题}
\newtheorem{lemma}{引理}
\newtheorem{corollary}{推论}
\newtheorem{remark}{注}
\newtheorem{condition}{条件}
\newtheorem{conclusion}{结论}
\newtheorem{assumption}{假设}
\newenvironment{solution}{\begin{proof}[\indent\bf 解]}{\end{proof}}%设置新环境
\CTEXsetup[format={\Large\bfseries}]{section}%加入字体
\geometry{a4paper,scale=0.8}%设置文档格式
\tikzstyle{file} = [rectangle, rounded corners, minimum width = 3cm, minimum height=1.2cm ,text centered, draw = black]%设置流程图
\tikzstyle{dots} = [rectangle, rounded corners, minimum width = 1.5cm, minimum height=2cm ,text centered, draw = black,text width=3cm]%设置流程图
\tikzstyle{arrow} = [->,>=stealth]%设置流程图
\usetikzlibrary{arrows, decorations.pathmorphing, backgrounds, positioning, fit, petri, automata}%使用流程图元素
\bibliographystyle{unsrt}%设置引用格式

\title{Homework4}
\author{作者:罗文杰\\专业: 计算机科学与技术\\学号: 3210102456}
\date{}

\begin{document}
\maketitle

\section*{4.3}
The program counter's value is the address of the next instruction, not the count of the sort. So the name 'Instruction Pointer' is more appropriate for it.

\section*{4.8}
(a) Since $2^8=256$ , it needs 8 bits to represent the OPCODE.

(b) Since $2^7=128$  , it needs 7 bits to represent the DR.

(c) The maximun number of UNUSED bits is 3 bits.

\section*{4.9}
The another is that making the MAR loaded with the contents of the PC, and incrementing the PC.

\section*{4.19}
(a) MAR:010 MDR:01010000

(b) MAR:001 MDR:00111001

\section*{5.1}
(a) ADD: operate; register addressing for destination and register or immediate addressing.

(b) JMP: control; register addressing.

(c) LEA: data movement; immediate addressing.

(d) NOT: operate; register addressing.

\section*{5.4}
(a)8bits.

(b)since the range of offset is 40, so the bits needed is 6.

(c)PC counter is incremented to 4, so the offset is 6.

\section*{5.9}
(a)it's not the same as NOP since it can sets the CC's.

(b)since the PC is incremented by 2 after the instruction(1 by uncondition), so the instruction is not the same as NOP.

(c)it's the same as NOP.

\bibliography{reference}
\end{document}
