\documentclass[20pt]{ctexart}
\usepackage{graphicx}
\usepackage{amsmath}
\usepackage{url}
\usepackage{subfigure}
\usepackage{float}
\usepackage{lmodern}
\usepackage{xeCJK}
\usepackage{tikz}
\usepackage{enumitem}
\usepackage{verbatimbox}
\usepackage{cite}
\usepackage{amsfonts}
\usepackage{amsthm}
\usepackage{geometry}
\usepackage{verbatimbox}
\usepackage{caption}
\usepackage{listings}
\usepackage[ruled,linesnumbered]{algorithm2e}

%设置新环境
\newtheorem{example}{例}             
\newtheorem{theorem}{定理}[section] 
\newtheorem{definition}{定义}[section]
\newtheorem{property}{性质}
\newtheorem{proposition}{命题}
\newtheorem{lemma}{引理}
\newtheorem{corollary}{推论}
\newtheorem{remark}{注}
\newtheorem{condition}{条件}
\newtheorem{conclusion}{结论}
\newtheorem{assumption}{假设}
\newenvironment{solution}{\begin{proof}[\indent\bf 解]}{\end{proof}}%设置新环境
\CTEXsetup[format={\Large\bfseries}]{section}%加入字体
\geometry{a4paper,scale=0.8}%设置文档格式
\tikzstyle{file} = [rectangle, rounded corners, minimum width = 3cm, minimum height=1.2cm ,text centered, draw = black]%设置流程图
\tikzstyle{dots} = [rectangle, rounded corners, minimum width = 1.5cm, minimum height=2cm ,text centered, draw = black,text width=3cm]%设置流程图
\tikzstyle{arrow} = [->,>=stealth]%设置流程图
\usetikzlibrary{arrows, decorations.pathmorphing, backgrounds, positioning, fit, petri, automata}%使用流程图元素
\bibliographystyle{unsrt}%设置引用格式

\title{Homework6}
\author{作者:罗文杰\\专业: 计算机科学与技术\\学号: 3210102456}
\date{}

\begin{document}
\maketitle

\section*{6.3}
Since the x4001 may correctly indirect which mechine is busy, we CAN'T just add to x4001, while using OR operation to make it.

x3000 1010 000 000001111; LDI R0, S 

x3001 1010 001 000001101; LDI R1, I 

x3002 0101 010 010 1 00000; AND R2, R2, \#0 

x3003 0001 010 010 1 00001; ADD R2, R2, \#1 

x3004 0001 001 001 1 11111; Loop ADD R1, R1, \#-1 

x3005 0000 100 000000010; BRz Decrement 

x3006 0001 010 010 0 00 010; ADD R2, R2, R2 

x3007 0000 111 111111100; BRz Loop 

x3008 0001 001 010 1 00000; Decement ADD R1, R2, \#0 

x3009 1001 000 000 111111; NOT R0, R0 

x300a 1001 001 001 111111; NOT R1, R1 

x300b 0101 000 000 0 00001; AND R0, R0, R1 

x300c 1001 000 000 111111; NOT R0, R0 

x300d 1011 000000000001; STI R0, S 

x300e 1111 000000100101; TRAP x25 

x300f 0100000000000000; I .FILL x4000 

x300e 0100000000000001; S .FILL x4001

\section*{7.4}
\begin{table}[H]
    \centering
    \begin{tabular}{|c|c|}
        \hline
        Symbol & Address  \\
        \hline
        TEST & x301F \\
        \hline
        FINSH & x3027 \\
        \hline  
        SAVE3 & x3029 \\
        \hline          
        SAVE2 & x302A \\
        \hline
        \end{tabular}
        \end{table}

\section*{7.18}
(a) LDR  R3, R1, \#0

(b) NOT  R3, R3

(c) ADD  R3, R3, \#1

\section*{7.20}
In program(a), the value will be stored when the program runs, while program(b) will store the vaule once the object module is loaded into Memory. 

\bibliography{reference}
\end{document}
